\section{Описание}

Требуется написать реализацию алгоритма поразрядной сортировки.


Идея сортировки заключается в разбиении сортируемых элементов на разряды. Затем выполняется сортировка для каждого разряда. Алгоритм для сортировки разрядов может быть выбран любым, но для максимальной эффективности нужно использовать сортировки за линейное время. При этом, для строк подходит версия MSD(Most Significant Digit) - сортировка начинается от самого старшего разряда. Для чисел же нужно использовать LSD-версию(Least Significant Digit). Для выполнения данной лабораторной работы была выбрана устойчивая сортировка подсчетом. В данной версии, сортировка начинается от самого младшего разряда.

{\bf Свойства сортировки подсчетом:}

	-- Не является сортировкой сравнением: ни одна пара
элементов не сравнивается друг с другом

	-- Линейная (вернее, $O(k + n)$, но при $k = O(n)$ время
выполнения $O(n)$)

	-- Устойчивая (стабильная)
	
	-- Не используются обмены(swap)
	
	-- Требует дополнительную память под массивы $C$ и $B$
размером k и n соответственно

\textbf{Теорема 1 (О времени работы программы)}

\textit{ Для $n$ $b$-битовых чисел и натурального числа $r$ $\leq$ $b$ (цифры из
$r$ битов) алгоритм Radix-Sort выполнит сортировку за время $\Theta(\dfrac{b}{r}(n + 2 ^r))$}


Тем самым, для $b < [log(n)]$, то асимптотически оптимален перебор $r = b$, иначе $r = [log(n)]$

r -- количество разрядов в самом длинном ключе.
b -- разрядность : количество значений разряда ключа

{\bf Описание работы программы}


	Сначала происходит ввод данных. Структура $TElement$ имеет 2 поля - для ключа и значения. Так как заранее неизвестно, сколько элементов будет обрабатываться, то исходный массив динамически расширяется. Как только встречается символ EOF(все данные введены), ввод считается завершенным и массив передается функции $sort$, которая сортирует данные. Так же, в функцию передается количество введеных элементов. Затем происходит печать того же массива, но уже отсортированного.
Функция $sort$ работает следующим образом: Нам заранее известен алфавит входных данных -- 16-ричные числа. Известна и длина каждой строки - 32. Создадим временный массив $A$ c типом $TElement$ и таким же размером как и исходный массив данных. Начинаем сортировку с конца строк. На каждом шагу(а всего их 32) заполняем массив $C$ из 16 элементов. Так мы узнаем сколько раз какой элемент алфавита был встречен.
	Далее начинаем использовать массив $A$, который является как-бы "снимком" исходного массива на данной итерации. Он нужен чтобы никакие данные при перемещении элементов не пропали. Благодаря массиву $C$, мы знаем какой элемент должен находится на определенном месте. Осталось только перезаписать данные в исходном массиве и начать следующую итерацию.
\pagebreak

\section{Исходный код}

1. $main.cpp$
\begin{lstlisting}[language=C]
#include <iostream>
#include <cstdio>
#include <cstdlib>
#include <cstring>
typedef struct TElement TElement;
struct TElement
{
    char Buffer[33];
    unsigned long long int n;
};
void sort(TElement*, int& amount_of_elems);
int main(int argc, char *argv[])
{
    int size_of_array = 1; // default value
    int amount_of_elems = 0;
    TElement* array = new TElement[size_of_array]; // array of elems
    while(1)
    {
        char ch = getchar();
        if(ch == EOF || ch == '\0')
            break;
        else
            ungetc(ch, stdin);
        if(amount_of_elems == size_of_array)
        {
            TElement* tmp_buffer = new TElement[size_of_array];
            memcpy(tmp_buffer, array, size_of_array*sizeof(TElement));
            array = new TElement[size_of_array * 2];
            memcpy(array, tmp_buffer, size_of_array*sizeof(TElement));
            size_of_array *= 2;
            delete[] tmp_buffer;
        }
        scanf("%s", array[amount_of_elems].Buffer);
        scanf("%llu", &array[amount_of_elems].n);
        ++amount_of_elems;
        getchar();
    }
    sort(array, amount_of_elems);
    for(int i = 0; i < amount_of_elems; ++i)
    {
        printf("%s\t", array[i].Buffer);
        printf("%llu\n", array[i].n);
    }
    delete[] array;
    return 0;
}

void sort(TElement* array, int& amount_of_elems)
{
    const int radix = 16;
    const int strLen = 32;
    TElement* A = new TElement [amount_of_elems];
    for(int digit = strLen - 1; digit >= 0; --digit)
    {
        int C[radix] = {0};
        for(int i = 0; i < amount_of_elems; ++i)
        {
            A[i] = array[i];
            int c = array[i].Buffer[digit];
            if (c >= '0' && c <= '9') c = c - '0';
            else c = c - 'a' + 10; // According to ASCII table
            ++C[c];
        }
        for(int i = 1; i < radix; ++i)
            C[i] = C[i] + C[i - 1];
        for(int i = amount_of_elems - 1; i >= 0; --i)
        {
            TElement val = A[i];
            int c = A[i].Buffer[digit];
            if (c >= '0' && c <= '9') c = c - '0';
            else c = c - 'a' + 10; // According to ASCII table
            int position = C[c] - 1;
            array[position] = val;
            --C[c];
        }
    }
    delete[] A;
}
\end{lstlisting}

2. $test generator.py$

\begin{lstlisting}[language=Python]
import random
import string

MAX_KEY_LEN = 33
MAX_VALUE_LEN = 8

def generate_random_key():
    return "".join( [ random.choice( string.hexdigits.lower() ) for _ in range( 1, MAX_KEY_LEN ) ] )

def generate_random_value():
    return "".join( [ random.choice( string.digits ) for _ in range( 1, MAX_VALUE_LEN ) ] )

if __name__ == "__main__":
    for num in range(1, 2):
        values = list()
        output_filename = "tests/{:02d}.t".format( num )
        with open( output_filename, 'w') as output:
            for _ in range( 10000000 ):
                key = generate_random_key()
                value = generate_random_value()
                values.append( (key, value) )
                output.write( "{}\t{}\n".format(key, value) )
        output_filename = "tests/{:02d}.a".format( num )
        with open( output_filename, 'w') as output:
            values = sorted( values, key=lambda x: x[0] )
            for value in values:
                output.write( "{}\t{}\n".format(value[0], value[1]) )
\end{lstlisting}
\section{Консоль}
\begin{alltt}
alex$ g++ -std=c++11 -o da_1 -pedantic -Wall -Werror -Wno-sign-compare -Wno-long-long -lm main.cpp sort.cpp 
alex$ cat 01.t 
00000000000000000000000000000000 13207862122685464576
ffffffffffffffffffffffffffffffff 7670388314707853312
00000000000000000000000000000000 4588010303972900864
ffffffffffffffffffffffffffffffff 12992997081104908288
alex$ ./da_1 < 01.t
00000000000000000000000000000000 13207862122685464576
00000000000000000000000000000000 4588010303972900864
ffffffffffffffffffffffffffffffff 7670388314707853312
ffffffffffffffffffffffffffffffff 12992997081104908288
\end{alltt}
\pagebreak

