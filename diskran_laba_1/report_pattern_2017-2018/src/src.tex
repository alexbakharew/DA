\section{Описание}
Требуется написать реализацию алгоритма поразрядной сортировки.

Идея сортировки заключается в разбиении сортируемых элементов на разряды. Затем выполняется устойчивая сортировка подсчетом для каждого разряда. При этом, для строк подходит версия MSD(Most Significant Digit) - сортировка начинается от самого старшего разряда. Для чисел же нужно использовать LSD-версию(Least Significant Digit). В данной версии, сортировка начинается от самого младшего разряда.

\pagebreak

\section{Исходный код}

На каждой непустой строке входного файла располагается пара \enquote{ключ-значение}. Ключ представляет собой строку из 32 элементов. Значение - целое неотрицательное число. Для хранения такой пары создается динамический масссив \textit{array} из объектов класса \textit{Telement}. Внутри этого класса содержатся непосредственно сама пара, а также 2 указателя на ключ и значение.
Данная особенность класса необходима для корректной сортировки строк без их изменения. Будут меняться только ссылки на соответствующие ключи и значения.
\begin{lstlisting}[language=C]

	
\end{lstlisting}

В случае, если код не помещается на одну-две страницы $A4$, тогда следует сделать табличку следующего вида:
\begin{longtable}{|p{7.5cm}|p{7.5cm}|}
\hline
\rowcolor{lightgray}
\multicolumn{2}{|c|} {main.c}\\
\hline
void sort(struct KV \& B, struct KV \& Res, int max, int size)&Функция сортировки подсчётом\\
\hline
\rowcolor{lightgray}
\multicolumn{2}{|c|} {file1.c}\\
\hline
void function\_name()&Функция, \enquote{которая почти всегда работает, но неясно, что она делает}.\\
\hline
\end{longtable}
В этом случае структуры или классы должны быть полностью приведены в листинге (без реализации методов).
\begin{lstlisting}[language=C]
struct KV{
	int key;
	char value;
} KV;
\end{lstlisting}
\pagebreak

\section{Консоль}
\begin{alltt}
a.kukhticev$ gcc -pedantic -Wall -std=c99 -Werror -Wno-sign-compare -lm da10.c -o da10 --some_long_argument=true
a.kukhticev$ cat test1 
87	a
13	b
89	c
13	d
a.kukhticev$ ./da10 < test1 
13	b
13	d
87	a
89	c
\end{alltt}
\pagebreak

