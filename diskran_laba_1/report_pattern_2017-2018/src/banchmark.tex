\section{Тест производительности}

Тест производительности представляет из себя следующее: Производится замер времени для работы поразрядной сортировки и сравнивается со временем работы сортировки std::sort().

Тесты производятся на основании сгенерированых текстовых файлов при помощи скрипта на Python. В каждом файле имеется от 1 до $10^{17}$ входных элементов.

\begin{alltt}
alex$ bash benchmark.sh 
g++ -std=c++11 -o da_1 -pedantic -Wall -Werror -Wno-sign-compare -Wno-long-long -lm main.cpp sort.cpp 
Time for Radix Sort

real	0m20.225s
user	0m19.748s
sys		0m0.472s
Time for std::sort()

real	0m28.851s
user	0m27.836s
sys		0m1.000s
alex$
\end{alltt}
\textbf{Ниже представлен листинг теста производительности:}
$benchmark.cpp$
\begin{lstlisting}[language=C]
#include <iostream>
#include <cstdio>
#include <cstdlib>
#include <cstring>
#include <vector>
#include <algorithm>
class TElement
{
public:
    TElement(){}
    ~TElement(){}
    friend std::istream& operator>>(std::istream&, const TElement&);
    friend std::ostream& operator<<(std::ostream&, const TElement&);
    bool operator >(const TElement&);
    bool operator <(const TElement&);
    bool operator ==(const TElement&);
    unsigned long long int* GetN();
    char* GetBuff();
private:
    char Buffer[33];
    unsigned long long int N;
};
bool TElement::operator<(const TElement& n1)
{
    if(strncmp(n1.Buffer, this->Buffer, 33) < 0) return true;
    else return false;
}
bool TElement::operator>(const TElement& n1)
{
    if(strncmp(n1.Buffer, this->Buffer, 33) > 0) return true;
    else return false;
}
bool TElement::operator==(const TElement& n1)
{
    if(strncmp(n1.Buffer, this->Buffer, 33) == 0) return true;
    else return false;
}
std::istream& operator>>(std::istream& input, TElement& n)
{
    char* Buff = n.GetBuff();
    input>>Buff;
    input.get();
    auto N = n.GetN();
    input>>*N;
    //input.get(); Commented for \0 recognition
    return input;
}
std::ostream& operator<<(std::ostream& output, const TElement& n)
{
    output<<n.Buffer;
    output<<"\t";
    output<<n.N;
    return output;
}
unsigned long long int* TElement::GetN()
{
    return &N;
}
char* TElement::GetBuff()
{
    return Buffer;
}
int main()
{
    std::vector<TElement> array;
    while(!std::cin.eof())
    {
        //std::cin.unget();
        TElement tmp;
        std::cin>>tmp;
        array.push_back(tmp);
        // std::cout<<"OK\n";
    }
    std::sort(array.begin(), array.end());
    for(int i = 0; i < array.size() - 1; ++i)
    {
        std::cout<<array[i]<<std::endl;
    }
}
\end{lstlisting}

Как видно, поразрядная сортировка выиграла по времени у быстрой сортировки(std::sort()). Это связано с тем, что сложность первой сортировки - линейная, а у std::sort() она равна $O(NlogN)$. На небольших данных разница во времени не ощущается, но при работе с внушительными объемами, поразрядная сортировка значительно лучше и производительнее. Стоит заметить, что если заменить сортировку подсчетом(в нашем случае эта сортировка являеется внутренней для поразрядной) на другую, которая будет работать ассимптотически дольше, то и вся сортировка в целом будет работать дольше. 

\pagebreak
