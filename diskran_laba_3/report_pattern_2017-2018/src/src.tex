\section{Описание}
Для выполнения данной лабораторной работы я решил реализовать класс бинарного дерева. Я не стал использовать AVL-дерево из прошлой лабораторной работы, потому что она не совсем хорошо написана -- есть много недочетов при планировании и реализации класса. Все помарки я учел и решил закрепить результат, написав бинарное дерево. Для профилирования моей программы я выбрал 2 инструмента -- gprof и valgrind.

gprof -- Помогает отслеживать время работы каждой функции. Есть 2 способа наблюдения: таблица и граф вызовов. Таблица сообщает, какое количество времени в процентах работала та или иная функция, а так же показывает число ее вызовов. Граф вызовов показывает, какая функция из какой вызывалась. Таким образом, мы можем отследить самые долгоработающие участки программы и решить, стоит ли их переписывать для ускорения работы нашего приложения.

valgrind -- Отслеживание утечек памяти в программе. Часто бывают случаи, когда некоторый участок памяти выделен, но не освобожден. Если мы работаем над небольшим проектом, то утечка памяти не так уж и страшна, ведь операционная система сама почистит память при завершении работы программы. Но если мы пишем только какую-то часть приложения, то тут могут быть различные проблемы с непредвиденным расходом памяти, что негативно сказывается на быстродействии. Так же, при неправильной работе с указателями, возможна ситуация, когда мы обращаемся к невыделенному участку памяти для нашей программы. Скорее всего, программа успешно скомпилируется и даже будет работать на некоторых ЭВМ, но для других возможны критические ошибки. Данная утилита помогает сократить расходы памяти, выявить потери данных и нарушения прав доступа к участкам памяти.

\pagebreak

\section{Исходный код}
Ниже приведен отлаженный и рабочий код бинарного дерева.

\textbf{btree.h}
\begin{lstlisting}[language=C]
class TNode
{
public:
    TNode();
    ~TNode();
    long GetVal();
    void SetVal(long val);
 private:
     TNode* Left;
     TNode* Right;
     long Value;

friend class TBtree;
};

class TBtree
{
public:
    TBtree();
    ~TBtree();
    bool Insert(long value);
    bool Remove(long value);
    TNode* Find(long value);
private:
    void ClearTree(TNode*& nd);
    bool Insert_(TNode*& nd, long val);
    bool Remove_(TNode*& nd, long val);
    TNode* Find_(TNode*& nd, long val);
    TNode* Root;
};
\end{lstlisting}
\textbf{btree.cpp}
\begin{lstlisting}[language=C]
#include <iostream>
#include <cstdlib>
#include "bin_tree.h"
TNode::TNode(){}
TNode::~TNode(){}
void TNode::SetVal(long val)
{
    Value = val;
}
long TNode::GetVal()
{
    return Value;
}
TBtree::TBtree()
{
    Root = nullptr;
}
void TBtree::ClearTree(TNode*& nd)
{
    if(nd == nullptr)
        return;
    if(nd->Left != nullptr)
        ClearTree(nd->Left);
    else if(nd->Right != nullptr)
        ClearTree(nd->Right);
    delete nd;
    nd = nullptr;
    return;
}

TBtree::~TBtree()
{
    ClearTree(Root);
}
bool TBtree::Insert(long val)
{
    if(Insert_(Root, val))
        return true;
    else return false;
}
TNode* TBtree::Find(long val)
{
    return Find_(Root, val);
}
TNode* TBtree::Find_(TNode*& nd, long val)
{
    if(nd == nullptr)
        return nullptr;
    if(nd->Value > val)
        Find_(nd->Left, val);
    else if(nd->Value < val)
        Find_(nd->Right, val);
    else return nd;
}
bool TBtree::Insert_(TNode*& nd, long val)
{
    if(nd == nullptr)
    {
        nd = new TNode;
        if(nd == nullptr)
            return false; // Not enough memory
        nd->Left = nd->Right = nullptr;
        nd->Value = val;
        return true;
    }
    else if(nd->Value > val)
        return Insert_(nd->Left, val);
    else if(nd->Value < val)
        return Insert_(nd->Right, val);
    return false; // item already added
}
\end{lstlisting}
\textbf{main.cpp}
\begin{lstlisting}[language=C]
#include <iostream>
#include <cstdlib>
#include <random>
#include "bin_tree.h"
int main()
{
    TBtree tree;
    std::random_device rd;  //Will be used to obtain a seed for the random number engine
    std::mt19937 gen(rd()); //Standard mersenne_twister_engine seeded with rd()
    std::uniform_int_distribution<> dis(-5000, 5000);
    for(int i = 0; i < 10000000; ++i)
        tree.Insert(dis(rd));
    for(int i = 0; i < 10000000; ++i)
        tree.Find(dis(rd)); 
    return 0;
}
\end{lstlisting}
\pagebreak

Исследуем данную программу на быстродействие и утечки памяти при помощи утилит valgrind и gprof
\textbf{•}
\begin{alltt}
alex$make
g++ -std=c++11 -pg -pedantic -Wall main.cpp bin_tree.cpp
alex$./a.out < 01.t
alex$gprof -D -b -a ./a.out gmon.out
Flat profile:

Each sample counts as 0.01 seconds.
  %   cumulative   self              self     total           
 time   seconds   seconds    calls   s/call   s/call  name    
 47.27      1.61     1.61 10000000     0.00     0.00  TBtree::Find_(TNode*&, long)
 40.77      2.99     1.39 10000001     0.00     0.00  TBtree::Insert_(TNode*&, long)
  6.50      3.21     0.22        1     0.22     3.42  main
  3.25      3.32     0.11 10000000     0.00     0.00  TBtree::Find(long)
  2.36      3.41     0.08 10000000     0.00     0.00  TBtree::Insert(long)
  0.30      3.42     0.01        1     0.01     0.01  TBtree::~TBtree()
  0.00      3.42     0.00    20001     0.00     0.00  TNode::TNode()
  0.00      3.42     0.00        9     0.00     0.00  TNode::~TNode()
  0.00      3.42     0.00        1     0.00     0.00  TBtree::ClearTree(TNode*&)
  0.00      3.42     0.00        1     0.00     0.00  TBtree::TBtree()
\end{alltt}
\section{Консоль}
\begin{alltt}
a.kukhticev$ gcc -pedantic -Wall -std=c99 -Werror -Wno-sign-compare -lm da10.c -o da10 --some_long_argument=true
a.kukhticev$ cat test1 
87	a
13	b
89	c
13	d
a.kukhticev$ ./da10 < test1 
13	b
13	d
87	a
89	c
\end{alltt}
\pagebreak

