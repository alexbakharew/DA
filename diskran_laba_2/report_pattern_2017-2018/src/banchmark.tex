\section{Тест производительности}
Сравним работу АВЛ-дерева и контейнера map. Ниже приведены листинги генератора тестов и использования std::map.

benchmark.cpp
\begin{lstlisting}[language=C]
void ToLowerCase(std::string& str)
{
    for(size_t i = 0; i != str.size(); ++i)
    {
        if(str[i] >= 'A' && str[i] <= 'Z')
            str[i] += 32;
    }
}
int main()
{
    std::map<std::string, unsigned long long int> lib;
    char command;
    std::string str;
    unsigned long long int val;
    std::cin.get(command);
    while(!std::cin.eof())
    {
        if(command == '+') // + akiufh 34
        {
            std::cin.get();// skip space
            std::cin>>str;
            ToLowerCase(str);
            std::cin.get();// skip space
            std::cin>>val;
            std::cin.get();// skip space
            auto it = lib.find(str);
            if((*it).first != str)// elem is not in lib
            {
                lib.insert(std::pair<std::string, unsigned long long int>(str, val));            
                std::cout<<"OK\n";
            }
            else
            {
                std::cout<<"Exist\n";
            }
        }
        else if(command == '-') // - a
        {
            std::cin.get();// skip space
            std::cin>>str;
            ToLowerCase(str);            
            std::cin.get();// skip space            
            auto it = lib.find(str);
            if((*it).first == str) //elem finded
            {
                lib.erase(it);
                std::cout<<"OK\n";
            }
            else
            {
                std::cout<<"NoSuchWord\n";
            }
        }
        else
        {
            std::cin.unget();
            std::cin>>str;
            ToLowerCase(str);        
            std::cin.get();// skip space
            auto it = lib.find(str);
            if((*it).first == str)
            {
                std::cout<<"OK: "<<(*it).second<<"\n";
            }
            else
            {
                std::cout<<"NoSuchWord\n";
            }
        }
        std::cin.get(command);
    } 
    return 0;
}
\end{lstlisting}
test.py
\begin{lstlisting}[language=Python]
import sys
import random
import string

def get_random_key():
    return random.choice( string.ascii_letters )

actions = ["+", "?", "-"]
test_file_name = "tests/01"
output_file = open("tests/01.t", 'w')
answer_file = open("tests/01.a", 'w')
for _ in range(2000):
    keys = dict()
    for _ in range(3000):
            action = random.choice( actions )
            if action == "+":
                key_1 = get_random_key()
                key_2 = get_random_key()
                key_3 = get_random_key()
                key_4 = get_random_key()
                key = key_1 + key_2 + key_3 + key_4
                value = random.randint( 1, 1000 )
                output_file.write("+ {0} {1}\n".format( key, value ))
                key = key.lower()
                answer = "Exist"
                if key not in keys:
                    answer = "OK"
                    keys[key] = value
                answer_file.write( "{0}\n".format( answer ) )
            elif action == "?":
                search_exist_element = random.choice( [ True, False ] )
                key = random.choice( [ key for key in keys.keys() ]) if search_exist_element and len( keys.keys() ) > 0 else get_random_key()
                output_file.write( "{0}\n".format( key ) )
                key = key.lower()
                if key in keys:
                    answer = "OK: {0}".format( keys[key] )
                else:
                    answer = "NoSuchWord"
                answer_file.write( "{0}\n".format( answer ) )
            elif action == "-":
                delete_exist_element = random.choice( [ True, False ] )
                key = random.choice( [ key for key in keys.keys() ] ) if search_exist_element and len( keys.keys() ) > 0 else get_random_key()
                output_file.write("- {0}\n".format(key))
                key = key.lower()
                if key in keys:
                    answer = "OK"
                    keys.pop(key)
                else:
                    answer = "NoSuchWord"
                answer_file.write( "{0}\n".format(answer))
\end{lstlisting}
benchmark.sh
\begin{lstlisting}[language=Bash]
#! /bin/bash
make
echo "Time for AVL-tree"
time ./*2 < tests/01.t > tmp
g++ -std=c++14 benchmark.cpp
echo "Time for MAP from STL"
time ./a.out < tests/01.t > tmp
\end{lstlisting}
\pagebreak
\textbf{Консоль}
\begin{alltt}
alex\$ bash bench*sh
g++ -std=c++11 -o diskran_laba_2 -pedantic -Wall -Werror -Wno-sign-compare -Wno-long-long -lm main.cpp avl_tree.cpp
Time for AVL-tree

real    0m6.525s
user    0m6.332s
sys     0m0.108s
Time for MAP from STL

real    0m26.850s
user    0m16.212s
sys     0m10.632s
alex\$
\end{alltt}
Как видно из теста производительности, AVL-дерево выигрывает по времени у std::map почти в 4-5 раз. Для теста были сгенерированы входные файлы, 6000000 строк в каждом. В них производится добавление, удаление и поиск элементов. 
В основе реализиции std::map лежит красно-черное дерево. Балансировка этого дерева происходит за $O(1)$, в то время как в AVL-дереве за $O(logn)$. Производительность работы программы зависит от ее использования. Если чаще всего нам придется искать элементы, то можно использовать AVL-дерево. Если же предвидятся частые вставки и удаления, то лучше использовать красно-черное дерево, так как балансировка будет происходить быстрее, а соответственно, быстрее будет работать программа.
\pagebreak
