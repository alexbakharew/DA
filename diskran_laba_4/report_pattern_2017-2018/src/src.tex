\section{Описание}

Алгоритм Кнута — Морриса — Пратта (КМП-алгоритм) — эффективный алгоритм, осуществляющий поиск подстроки в строке. Время работы алгоритма линейно зависит от объёма входных данных, то есть разработать асимптотически более эффективный алгоритм невозможно. Так же хочется отметить, что никакой элемент ни строки, ни образца не сравниваются больше одного раза!

Основным отличием алгоритма Кнутта - Морриса - Пратта от алгоритма прямого
поиска заключается в том, что сдвиг подстроки выполняется не на один
символ на каждом шаге алгоритма, а на некоторое переменное количе-
ство символов. Следовательно, перед тем, как осуществлять очередной
сдвиг, необходимо определить величину сдвига. Для повышения эффек-
тивности алгоритма необходимо, чтобы сдвиг на каждом шаге был бы
как можно большим.


\pagebreak

\section{Исходный код}

\textbf{main.cpp}
\begin{lstlisting}[language=C]
#include <iostream>
#include <vector>
#include <string>
std::vector<std::string> PatternParse() 
{
    char c;
    std::string buf;
    std::vector<std::string> pattern;
    while (true) 
    {
        c = getchar_unlocked();
        if (c == ' ') 
        {
            if (!buf.empty())
                pattern.push_back(buf);
            buf.clear();
        }
        else if (c == '\n' || c == EOF) 
        {
            if (!buf.empty())
                pattern.push_back(buf);
            break;
        }
        else
            buf.push_back(c);
    }
    return pattern;
}
std::vector<int> Zfunc(std::vector<std::string>& pat) 
{
    const unsigned long n = pat.size();
    std::vector<int> zArray(n);
    for (int i = 1, l = 0, r = 0; i < n; ++i) 
    {
        if (i <= r)
            zArray[i] = std::min(r - i + 1, zArray[i - l]);
        while (i + zArray[i] < n && pat[zArray[i]] == pat[i + zArray[i]])
            ++zArray[i];
        if (i + zArray[i] - 1 > r) 
        {
            l = i; 
            r = i + zArray[i] - 1;
        }
    }
    return zArray;
}
std::vector<int> CalcSP(std::vector<int>& zArray, std::vector<std::string>& pat) 
{
    const unsigned long n = pat.size();
    std::vector<int> spArray(n);
    for (unsigned long j = n - 1; j > 0; --j) 
    {
        unsigned long i = j + zArray[j] - 1;
        spArray[i] = zArray[j];
    }
    return spArray;
}
std::vector<int> FailFunction(std::vector<std::string>& pattern) 
{
    unsigned long n = pattern.size();
    std::vector<int> zArray = Zfunc(pattern);
    std::vector<int> spArray = CalcSP(zArray, pattern);
    std::vector<int> f(n + 1);
    f[0] = 0;
    for (int k = 1; k < n; ++k) 
        f[k] = spArray[k - 1];
    f[n] = spArray[n - 1];
    return f;
}
void Kmp(std::vector<std::string>& pat) 
{
    std::vector<std::pair<std::pair<int, int>, std::string>> text;
    char c = '$';                                       // default value
    bool wordAdded = true;                              // True if last action was to add the word into the vector
    std::pair<std::pair<int, int>, std::string> temp;   // current word
    const unsigned long n = pat.size();                 // Pattern size
    temp.first.first = 1;                               // Row counter
    temp.first.second = 1;                              // Word counter
    int p = 0;                                          // Pattern pointer
    int t = 0;                                          // Text pointer
    std::vector<int> f = FailFunction(pat);
    do 
    {
        while (text.size() < pat.size() && c != EOF) 
        {
            c = getchar_unlocked();
            if (c == '\n') 
            {
                if (!wordAdded) 
                {
                    text.push_back(temp);
                    temp.second.clear();
                    wordAdded = true;
                }
                ++temp.first.second;
                temp.first.first = 1;
            } 
            else if (c == ' ' || c == '\t' || c == EOF) 
            {
                //skip spaces
                if (wordAdded)
                    continue;
                text.push_back(temp);
                temp.second.clear();
                ++temp.first.first;
                wordAdded = true;
            } 
            else 
            {
                if (temp.second.front() == '0')                // Deletes unnecessary zeros in the number
                    temp.second.erase(temp.second.begin());
                temp.second.push_back(c);
                wordAdded = false;
            }
        }
        if (text.size() < pat.size())// There are no possible matches
            return;
        while (p < n && pat[p] == text[t].second ) 
        { // We are checking if pattern suits or not
            ++p;
            ++t;
        }
        if (p == n) 
        { // We found match
            printf("%d, %d\n", text[0].first.second, text[0].first.first);
        }
        if (p == 0) 
            ++t;
        p = f[p];
        text.erase(text.begin(), text.begin() + t - p);
        t = p;
    } 
    while (c != EOF);
}
int main()
{
    std::vector<std::string> pattern = PatternParse();
    Kmp(pattern);
    return 0;
}

\end{lstlisting}

\pagebreak

\section{Консоль}
\begin{alltt}
alex$make
g++ -o da_4 -pg -std=c++11 -pedantic -Wall -Werror -Wno-sign-compare \\ -Wno-long-long main.cpp -lm
alex$cat 01.t
11 45 11 45 90
0011 45 011 0045 11 45 90    11
45 11 45 90
alex$./*4 < 01.t
1, 3
1, 8
alex$cat 02.t
80

80
00080  80 

90
alex$./*4 < 02.t
2, 1
3, 1
3, 2
alex$
\end{alltt}
\pagebreak

