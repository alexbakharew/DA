\section{Описание}

В общем смысле граф представляется как множество вершин (узлов), соединён-
ных рёбрами. В строгом определении графом называется такая пара множеств $G =
(V, E)$, где $V$ есть подмножество любого счётного множества, а $E$ — подмножество $V * V$ .

\textbf{Виды графов:}

$\bullet$ Неориентированный

$\bullet$ Ориентированный

Отношение связности, компоненты связности:
Две вершины $u$ и $v$ называются связаными, если в графе $G$ существует путь из $u$ в
$v$ .

\textbf{Путь, цепь, цикл:}

$\bullet$ Маршрут в графе — это чередующаяся последовательность вершин и рёбер, в кото-
рой любые два соседних элемента инцидентны. Если $v_{0} = v_{k}$ , то маршрут замкнут,
иначе открыт.

$\bullet$ Простая цепь — маршрут, в котором все вершины различны.

$\bullet$Простой граф — граф, в котором нет кратных рёбер и петель.

$\bullet$ Простой путь — путь, все рёбра которого попарно различны. Другими словами,
простой путь не проходит дважды через одно ребро.

$\bullet$ Простой цикл — цикл, не проходящий дважды через одну вершину.

$\bullet$ Псевдограф — граф, который может содержать петли и/или кратные рёбра.

$\bullet$ Путь — последовательность рёбер (в неориентированном графе) и/или дуг (в ори-
ентированном графе), такая, что конец одной дуги (ребра) является началом другой
дуги (ребра). Или последовательность вершин и дуг (рёбер), в которой каждый эле-
мент инцидентен предыдущему и последующему. Может рассматриваться как част-
ный случай маршрута.

\pagebreak

\section{Исходный код}
Для поиска компонент связности я использовал алгоритм поиска в ширину. Для каждой вершины, если мы еще там не были, запускался BFS. Одновременно записывался путь, по которому мы идем. Если обход заканчивался, значит мы нашли компоненту связности. 

\textbf{main.cpp}
\begin{lstlisting}[language=C]
#include "TGraph.h"

int main() {
    TGraph graph;
    std::cin >> graph;
    std::vector<std::set<std::size_t> > answer = graph.FindComponents();

    for (int i = 0; i < answer.size(); i++) {
        std::set<std::size_t>::iterator it = answer[i].end();
        it--;
        for (std::set<std::size_t>::iterator j = answer[i].begin(); j != it; j++) {
            std::cout << *j+1 << " ";
        }
        
        std::cout <<*it+1 <<std::endl;
    }
    return 0;
}
\end{lstlisting}

\textbf{TGraph.cpp}
\begin{lstlisting}[language=C]
void TGraph::DFS(std::size_t start, std::vector<bool>& used_vertex) {
    used_vertex[start] = true;
    components[components.size() - 1].insert(start);
    for (std::set<std::size_t>::iterator it = data[start].begin();
                                    it != data[start].end(); ++it) {

        if (used_vertex[*it] == false)
            DFS(*it, used_vertex);
    }
}
std::vector<std::set<std::size_t> >  TGraph::FindComponents() {
    std::size_t size = data.size();
    std::vector<bool> used_vertex(size, false);

    for (std::size_t i = 0; i < size; ++i)
        if (used_vertex[i] == false) {
            components.emplace_back();
            DFS(i, used_vertex);

        }
    return components;
}

TGraph::TGraph() = default;

std::istream &operator>>(std::istream &is, TGraph &graph){
    std::size_t n, m;
    is >> n >> m;
    graph.data.resize(n);
    for (int i = 0; i < m; i++) {
        std::size_t x, y;
        is >> x >> y;
        graph.data[x-1].insert(y-1);
        graph.data[y-1].insert(x-1);
    }
    return is;
}
\end{lstlisting}

\textbf{TGraph.h}
\begin{lstlisting}[language=C]
#include <vector>
#include <istream>
#include <set>

class TGraph {
private:
    std::vector<std::set<std::size_t> > data;
    std::vector<std::set<std::size_t> > components;
    std::size_t size;
public:
    explicit TGraph();

    std::vector<std::set<std::size_t> > FindComponents();
    void DFS(std::size_t start, std::vector<bool>& used_vertex);
    friend std::istream &operator>>(std::istream &is, TGraph &graph);
    ~TGraph() {}
};
\end{lstlisting}
\pagebreak

\section{Консоль}
\begin{alltt}
alex$ make
g++ -std=c++11 -pedantic -Werror -Wno-sign-compare -Wno-long-long -lm -c TGraph.cpp
g++ -std=c++11 -pedantic -Werror -Wno-sign-compare -Wno-long-long -lm -o lab9 main.cpp TGraph.o
alex$ ./lab9 
5 4
1 2
2 3
1 3
4 5
1 2 3
4 5
\end{alltt}
\pagebreak

