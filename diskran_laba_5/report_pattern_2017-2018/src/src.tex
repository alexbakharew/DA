\section{Описание}

\subsection*{Суффиксное дерево}

Суффиксное дерево $T$ для $m$-символьной строки $S$:

$\bullet$ Ориентированное дерево, имеющее
ровно $m$ листьев, пронумерованных от 1 до $m$.

$\bullet$ Каждая внутренняя вершина, отличная
от корня, имеет не меньше двух детей.

$\bullet$ Каждая дуга помечена непустой
подстрокой строки $S$ (дуговая метка).

$\bullet$ Никакие две дуги, выходящие из одной
вершины, не могут иметь меток,
начинающихся с одинаковых символов.

$\bullet$ Для каждого листа $i$ конкатенация
меток от корня составляет $S[i..m]$



Наивный алгоритм строит суффиксное дерево за $O(m^2)$, однако его нельзя улучшить до линейного времени. Эско Укконен произвел несколько разумных и вполне простых трюков над алгоритмом, который работал за $O(m^3)$. Теперь мы можем строить суффиксное дерево за $O(m)$.
\\

\textbf{Общее описание алгоритма Укконена:}

-- Последовательно строим неявные деревья $T_i$ для каждого
префикса $S[1..i]$.

-- Настоящее суффиксное дерево $T$ можно получить из $T_m$
построив следующее неявное дерево для строки с терминальным символом.

\textbf{Первое ускорение: Суффиксные связи}

\textbf{Определение} Пусть $x\alpha$ обозначает произвольную строку, где $x$ -- её первый символ,а $\alpha$ -- оставшаяся подстрока. Если для внутренней вершины $u$ с путевой меткой $x\alpha$ существует другая вершина $s(u)$ с путевой меткой $\alpha$, то указатель из $u$ в $s(u)$ назы-
вается \textbf{суффиксной связью}.

\textbf{Второе ускорение: Сжатие суффиксных меток}

В каждом ребре будет хранится не только один символ, а целая подстрока. Причем будем хранить ее не явно, а только координаты начала и конца. Так же, для всех листьев сделаем общий счетчик $end$. При добавлении новой буквы увеличиваем его на 1.

\subsection*{Поиск образца в тексте}
$\bullet$ Строится суффиксное дерево для текста.

$\bullet$ Ищется путь,
совпадающий с образцом.
Если такого пути нет, то
образец в текст не входит.

$\bullet$ Если путь есть, то все
листья поддерева --
вхождения.

\subsection*{Построение суффиксного массива} 

$\bullet$ Для текста T построить суффиксное дерево T.

$\bullet$ Обойти дерево T в глубину таким образом, что первыми
проходятся дуги, чьи метки меньше остальных в
лексикографическом смысле.

$\bullet$ Если дуги хранятся в порядке возрастаний первых
символов меток, то такой обход будет натуральным

$\bullet$ Суффиксный массив -- просто список посещений листьев
при таком обходе.

$\bullet$ Тем самым, суффиксный массив строится за время $O(m)$.

\subsection*{Поиск в суффиксном массиве} 

$\bullet$ Составляем обычный массив суффиксов данной строки длины $m$

$\bullet$ Сортируем массив в лексикографическом порядке. Теперь мы готовы искать вхождения образцов.

$\bullet$ Ищем вхождение при помощи бинарного поиска. Если нашли, то как минимум одно
вхождение есть. Нужно проверить, нет ли еще. Начинаем "расширять границы" найденного шаблона. Двигаемся влево и вправо, пока шаблон является префиксом для других суффиксов строки.

$\bullet$ Выводим индексы совпадений

\subsection*{Работа программы}

$\blacktriangleright$ Производим инициализацию суффиксного дерева. Передаем данные об алфавите и терминальном символе.

$\blacktriangleright$ Считываем текст. Производим вставку каждой буквы в суффиксное дерево. После этого имеем неявное суффиксное дерево. Вставляем терминальный символ, тем самым получая явное суффиксное дерево.

$\blacktriangleright$ Выполняем преобразование нашего дерева в суффиксный массив. Теперь мы готовы искать подстроки.

$\blacktriangleright$ Считываем по очереди все паттерны. Ищем их в суффиксном массиве и получаем вектор вхождений.
\pagebreak

\section{Исходный код}
suffix\_tree.h
\begin{lstlisting}[language=C]
#define MAX_LENGTH std::numeric_limits<std::size_t>::max()
class TNode 
{
public:
    using link = std::map<char, TNode *>;
    link edges;
    TNode *parent;
    TNode *suffixLink;
    std::size_t begin;
    std::size_t length;
    TNode(const std::string &, TNode * const &);
    TNode(const std::vector<char> &, const char &);
    TNode(const std::vector<char> &, const char &, TNode * const &);
    TNode(const std::string &, TNode * const &, const std::size_t &);
    virtual ~TNode();
};
class TSuffixTree;
class TSuffixArray;
class TSuffixArray
{
private:
    std::string text;
    std::vector<std::size_t> array;
public:
    explicit TSuffixArray(const TSuffixTree &);
    std::vector<std::size_t> Find(const std::string &pattern) const;
    virtual ~TSuffixArray();
};
class TSuffixTree
{
    std::string text;
    TNode *root;
    TNode *active_vertex;
    std::size_t activeLength;
    std::size_t activeCharIdx;
    void DFS(TNode * const &, std::vector<std::size_t> &, const std::size_t &) const;
    friend TSuffixArray::TSuffixArray(const TSuffixTree &);
public:
    explicit TSuffixTree(const std::vector<char> &, const char &);
    void PushBack(const char &ch);
    virtual ~TSuffixTree();
};
\end{lstlisting}
\pagebreak
\begin{longtable}{|p{7.5cm}|p{7.5cm}|}
\hline
\hline
\rowcolor{lightgray}
\multicolumn{2}{|c|} {suffix\_tree.cpp}\\
\hline

explicit TSuffixTree(const std::vector<char>\& , const char\&)&Конструктор суффиксного дерева. Для создания экземпляра класса необходимо передать в констуктор алфавит и терминирующий символ.\\
\hline
explicit TSuffixArray(const TSuffixTree \&)&Конструктор класса суффиксного массива. Массив строится на основании построенного суффиксного дерева.\\
\hline
void TSuffixTree::PushBack(const char\& new\_ch)&Функция добавления символа в суффиксное дерево в режиме реального времени.\\
\hline
void TSuffixTree::DFS(TNode\* const \&curr, std::vector<std::size\_t> \&result, const std::size\_t \&summary)&Преобразование суффиксного дерева в суффиксный массив при помощи алгоритма обхода в глубину.\\
\hline
std::vector<std::size\_t> TSuffixArray::Find(const std::string \&pattern)&Поиск подстроки в суффиксном массиве.\\
\hline
\end{longtable}
\pagebreak

\section{Консоль}
\begin{alltt}
alex$make
g++ suffix_tree.cpp main.cpp -std=c++11 -pedantic -Wall -Werror -Wno-sign-compare -Wno-long-long -lm -o diskran_laba_5
alex$ cat 01.t 
bbcabcbacdcabbdbabacbacaddaabcaaddacbccbdaacbdcaaddbabbddadabcbbbdadcaadacbaaada\\abbacbdddccbcabccdadabaccdacaddbbccd
bbac
bbca
aabc
dcab
dadd
dbab
dbac
bbba
acbd
bbba
ccab
bdad
dadd
bbca
alex$ ./*5 < 01.t 
1: 17
3: 49
4: 95
5: 82
6: 1
7: 27
8: 10
10: 15, 51
13: 43, 84
16: 65
18: 1
alex$
\end{alltt}
\pagebreak

