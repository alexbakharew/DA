\section{Тест производительности}
Поиск подстроки в строке при помощи суффиксного массива я сравнивал с наивным алгоритмом.
Ниже приведен листинг бенчмарка и генератора тестов на python:
benchmark.cpp
\begin{lstlisting}[language=C]
#include <cstdlib>
#include <iostream>
#include <string>
#include <vector>
std::vector<int> Find(const std::string& pattern, const std::string& text)
{
    std::vector<int> res;
    for(int i = 0; i < text.size(); ++i)
    {
        int k = i;
        for(int j = 0; j < pattern.size(); ++j)
        {
            if(text[k] == pattern[j])
                ++k;
            else
                break;
        }
        if(i - k == pattern.size() - 1)
            res.push_back(i);
    }
    return res;
}
int main()
{
    std::string text;
    std::string pattern;
    std::cin>>text;
    size_t n = 1;
    std::string buf;
    while(std::cin>>pattern)
    {
        std::vector<int> res = Find(pattern, text);
        if(!res.empty())
        {
            std::cout<<n<<": ";
            for(auto i : res)
                std::cout<<res[i]<<", ";
        }
        ++n;
    }
    return 0;
}
\end{lstlisting}
\pagebreak
Generator.py
\begin{lstlisting}[language=python]
import random
import sys
def GenText(l, alphabet):
    text = str()
    for i in range(l):
        text += random.choice(alphabet)
    return text

inputFile = open("01.t", "w")
outputFile = open("01.a", "w")
textLen = 1000000
alphabet = ('a', 'b', 'c', 'd')
text = str(GenText(textLen, alphabet))# Generate text
inputFile.write(text + '\n')
pCount = random.randint(1000, 2000)# Generate count of patterns
for i in range(1, pCount):
    pattern = str()
    for j in range(random.randint(20, 50)):
        pattern += random.choice(alphabet)
    inputFile.write(pattern + "\n")
    buf = str()
    res = text.find(pattern, 0, len(text))
    isMatch = False
    if res != -1:
        buf += str(i) + ": " #write num of pattern if we have at least one match
    while res != -1:
        if isMatch:
            buf += ", "
        isMatch = True
        buf += str(res)
        res = text.find(pattern, res + 1, len(text))
    if len(buf) != 0:
        outputFile.write(buf + "\n")
\end{lstlisting}
\pagebreak
\begin{alltt}
alex$python3 Gen*
alex$wc 01.t
   1120    1120 1004472 01.t
alex
alex$make
g++ suffix_tree.cpp main.cpp -std=c++11 -pedantic -Wall -Werror -Wno-sign-compare -Wno-long-long -lm -o diskran_laba_5
alex$g++ -std=c++11 bench*
alex$time ./a.out < 01.t > tmp

real    0m30.215s
user    0m30.204s
sys     0m0.000s
alex$time ./*5 < 01.t > tmp
real    0m2.946s
user    0m2.556s
sys     0m0.308s
\end{alltt}

Как видно из теста производительности, суффиксное дерево быстрее чем наивный алгоритм. Разница колоссальная. Текст состоит из большого количества символов, шаблоны много меньше по длине. Но если длина паттерна будет, к примеру, 4-5 символов, то разницы во времени работы наших алгоритмов не будет. Это связано с тем, что наивный алгоритм поиска хорошо показывает себя, когда нам надо найти в большом тексте маленький фрагмент. Время работы будет линейно зависеть от длины исходного текста(длину паттерна в этом случае можно принять за константу). Однако при работе с большими образцами, время работы этого алгоритма возрастет до квадрата от длины текста(в худшем случае), в то время как суффиксное дерево будет работать линейно от длины шаблона! Так что, по моему мнению, нет абсолютно универсального и удобного метода поиска подстроки в строке. Нужно смотреть по ситуации, что в данном случае проще, удобнее и самое главное быстрее.
\pagebreak
