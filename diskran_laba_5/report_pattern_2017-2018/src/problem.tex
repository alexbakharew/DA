\CWHeader{Лабораторная работа \textnumero 5}

\CWProblem{
Необходимо реализовать алгоритм Укконена построения суффиксного дерева за линейное время. Построив такое дерево для некоторых из входных строк, необходимо воспользоваться полученным суффиксным деревом для решения своего варианта задания.

Алфавит строк: строчные буквы латинского алфавита (т.е., от a до z).

{\bfseries Вариант задания : 2} Поиск с использованием суффиксного массива.
Найти в заранее известном тексте поступающие на вход образцы с использованием суффиксного массива.

{\bfseries Входные данные:} { \normalfont\ttfamily текст располагается на первой строке, затем, до конца файла, следуют строки с образцами.}


{\bfseries Выходные данные:} { \normalfont\ttfamily для каждого образца, найденного в тексте, нужно распечатать строчку, начинающуюся с последовательного номера этого образца и двоеточия, за которым, через запятую, нужно перечислить номера позиций, где встречается образец в порядке возрастания.}
}
\pagebreak