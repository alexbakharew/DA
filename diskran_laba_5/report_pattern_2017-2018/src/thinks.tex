\section{Выводы}
Выполнив пятую лабораторную работу по курсу \enquote{Дискретный анализ}, я познакомился с новым алгоритмом поиска подстрок в тексте -- суффиксное дерево. Он крайне эффективен, если нам заранее известен текст. Мы подготавливаем структуру дерева для эффективного поиска любого количества образцов. Время построения суффиксного дерева -- линейное. Время работы при поиске подстрок -- тоже линейное, зависящее от их длин. Существует несколько реализаций построений суффиксных деревьев. Однако самым простым и эффективным является алгоритм Укконена. Его большой плюс в том, что нам не нужно хранить строку с текстом целиком. Мы считываем символ за символом и строим наше дерево. Это так называемый алгоритм реального времени(online algorithm). Его можно использовать в большом спектре задач, где нужна интерактивность и минимальное время отклика. 
\pagebreak